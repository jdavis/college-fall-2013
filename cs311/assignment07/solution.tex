\documentclass{article}

\usepackage{fancyhdr}
\usepackage{lastpage}
\usepackage{extramarks}
\usepackage[usenames,dvipsnames]{color}
\usepackage{amsmath}
\usepackage{amsthm}
\usepackage{amsfonts}
\usepackage{changepage}
\usepackage{lineno}
\usepackage[plain]{algorithm}
\usepackage{algpseudocode}
\usepackage{tikz}
\usepackage{hyperref}

\usetikzlibrary{arrows}
\usetikzlibrary{positioning}

\topmargin=-0.45in
\evensidemargin=0in
\oddsidemargin=0in
\textwidth=6.5in
\textheight=9.0in
\headsep=0.25in

\linespread{1.1}

\pagestyle{fancy}
\lhead{\hmwkAuthorName}
\chead{\hmwkClass\ (\hmwkClassInstructor\ \hmwkClassTime): \hmwkTitle}
\rhead{\firstxmark}
\lfoot{\lastxmark}
\cfoot{}

\renewcommand\headrulewidth{0.4pt}
\renewcommand\footrulewidth{0.4pt}

\setlength{\floatsep}{100pt}
\renewcommand{\algorithmicrequire}{\textbf{Input:}}
\renewcommand{\algorithmicensure}{\textbf{Output:}}
\algrenewcomment[1]{\hfill // #1}

\setlength\parindent{0pt}

\hypersetup{colorlinks=true}

\newcommand{\enterProblemHeader}[1]{
    \nobreak\extramarks{}{Problem \arabic{#1} continued on next page\ldots}\nobreak{}
    \nobreak\extramarks{Problem \arabic{#1} (continued)}{Problem \arabic{#1} continued on next page\ldots}\nobreak{}
}

\newcommand{\exitProblemHeader}[1]{
    \nobreak\extramarks{Problem \arabic{#1} (continued)}{Problem \arabic{#1} continued on next page\ldots}\nobreak{}
    \stepcounter{#1}
    \nobreak\extramarks{Problem \arabic{#1}}{}\nobreak{}
}

\setcounter{secnumdepth}{0}
\newcounter{homeworkProblemCounter}
\setcounter{homeworkProblemCounter}{1}
\nobreak\extramarks{Problem \arabic{homeworkProblemCounter}}{}\nobreak{}

\newenvironment{homeworkProblem}{
    \section{Problem \arabic{homeworkProblemCounter}}
    \enterProblemHeader{homeworkProblemCounter}
}{
    \exitProblemHeader{homeworkProblemCounter}
}

\newcommand{\hmwkTitle}{Homework\ \#7}
\newcommand{\hmwkDueDate}{December 6, 2013 at 4:30pm}
\newcommand{\hmwkClass}{CS311}
\newcommand{\hmwkClassTime}{Section 3}
\newcommand{\hmwkClassInstructor}{Professor Lathrop}
\newcommand{\hmwkAuthorName}{Josh Davis}
\newcommand{\solution}{{\large \bfseries Solution}}

\title{
    \vspace{2in}
    \textmd{\textbf{\hmwkClass:\ \hmwkTitle}}\\
    \normalsize\vspace{0.1in}\small{Due\ on\ \hmwkDueDate}\\
    \vspace{0.1in}\large{\textit{\hmwkClassInstructor\ \hmwkClassTime}}
    \vspace{3in}
}

\author{\textbf{\hmwkAuthorName}}
\date{}

\newcommand{\alg}[1]{\textsc{\bfseries \footnotesize #1}}

\begin{document}

\maketitle

\pagebreak

\begin{homeworkProblem}
    Use simulated annealing to find approximate solutions to the optimization
    variant of \alg{No-Three-In-Line}.
    \\

    \textbf{Part A}

    Specify in pseudocode a fitness function that evaluates potential solutions
    to \alg{No-Three-In-Line}.
    \\

    \solution

    \begin{algorithm}[]
        \begin{algorithmic}[1]
            \Function{NoThreeInLine}{$a, b, c$}
                \Return{0}
            \EndFunction{}
        \end{algorithmic}
        \caption{Fitness algorithm}
    \end{algorithm}

    \textbf{Part B}

    Specify an appropriate set of moves for the \alg{No-Three-In-Line} problem.
    \\

    \solution

    \textbf{Part C}

    Discuss strengths and weaknesses of your fitness function and set of moves.
    Are potential solutions that the set of moves label as 'near' actually
    similar in a useful sense? Why or why not?  How does this affect the
    effectiveness of the simulated annealing algorithm?
    \\

    \solution
\end{homeworkProblem}

\pagebreak

\begin{homeworkProblem}
\end{homeworkProblem}

\pagebreak

\begin{homeworkProblem}
\end{homeworkProblem}

\pagebreak

\begin{homeworkProblem}
\end{homeworkProblem}

\pagebreak

\begin{homeworkProblem}
\end{homeworkProblem}

\pagebreak

\begin{homeworkProblem}
\end{homeworkProblem}

\end{document}
