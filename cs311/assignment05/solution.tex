\documentclass{article}

\usepackage{fancyhdr}
\usepackage{lastpage}
\usepackage{extramarks}
\usepackage[usenames,dvipsnames]{color}
\usepackage{amsmath}
\usepackage{amsthm}
\usepackage{amsfonts}
\usepackage{changepage}
\usepackage{lineno}
\usepackage{algorithm}
\usepackage{algpseudocode}

\topmargin=-0.45in
\evensidemargin=0in
\oddsidemargin=0in
\textwidth=6.5in
\textheight=9.0in
\headsep=0.25in

\linespread{1.1}

\pagestyle{fancy}
\lhead{\hmwkAuthorName}
\chead{\hmwkClass\ (\hmwkClassInstructor\ \hmwkClassTime): \hmwkTitle}
\rhead{\firstxmark}
\lfoot{\lastxmark}
\cfoot{}

\renewcommand\headrulewidth{0.4pt}
\renewcommand\footrulewidth{0.4pt}

\renewcommand{\algorithmicrequire}{\textbf{Input:}}
\renewcommand{\algorithmicensure}{\textbf{Output:}}

\setlength\parindent{0pt}

\newcommand{\enterProblemHeader}[1]{
    \nobreak\extramarks{}{Problem \arabic{#1} continued on next page\ldots}\nobreak{}
    \nobreak\extramarks{Problem \arabic{#1} (continued)}{Problem \arabic{#1} continued on next page\ldots}\nobreak{}
}

\newcommand{\exitProblemHeader}[1]{
    \nobreak\extramarks{Problem \arabic{#1} (continued)}{Problem \arabic{#1} continued on next page\ldots}\nobreak{}
    \stepcounter{#1}
    \nobreak\extramarks{Problem \arabic{#1}}{}\nobreak{}
}

\setcounter{secnumdepth}{0}
\newcounter{homeworkProblemCounter}
\setcounter{homeworkProblemCounter}{1}
\nobreak\extramarks{Problem \arabic{homeworkProblemCounter}}{}\nobreak{}

\newenvironment{homeworkProblem}{
    \section{Problem \arabic{homeworkProblemCounter}}
    \enterProblemHeader{homeworkProblemCounter}
}{
    \exitProblemHeader{homeworkProblemCounter}
}

\newcommand{\hmwkTitle}{Homework\ \#5}
\newcommand{\hmwkDueDate}{November 11, 2013 at 4:30pm}
\newcommand{\hmwkClass}{CS311}
\newcommand{\hmwkClassTime}{Section 3}
\newcommand{\hmwkClassInstructor}{Professor Lathrop}
\newcommand{\hmwkAuthorName}{Josh Davis}

\title{
    \vspace{2in}
    \textmd{\textbf{\hmwkClass:\ \hmwkTitle}}\\
    \normalsize\vspace{0.1in}\small{Due\ on\ \hmwkDueDate}\\
    \vspace{0.1in}\large{\textit{\hmwkClassInstructor\ \hmwkClassTime}}
    \vspace{3in}
}

\author{\textbf{\hmwkAuthorName}}
\date{}

\newcommand{\alg}[1]{\textsc{\bfseries \footnotesize #1}}

\begin{document}

\maketitle

\pagebreak

\begin{homeworkProblem}
    Give algorithms for the following operations.
    \\

    \textbf{Solution}

    Algorithms for each part are below.
    \\

    \textbf{Part A}

    Give an algorithm that multiples two degree-1 polynomials with only three
    multiply operations.
    \\

    \begin{algorithm}[]
        \begin{algorithmic}[1]
            \Function{MultiplySingleDegreePolynomials}{}
            \State{} \Return{$0$}
            \EndFunction{}
        \end{algorithmic}
        \caption{Multiply 1 degree polynomial}
    \end{algorithm}

    \textbf{Part B}

    Give a divide-and-conquer algorithm for multiplying two polynomials of degree \(n\). Prove
    using the master theorem that your algorithm runs in \(\Theta(n^{\log_2 3})\). You may assume
    that \(n + 1\) is a power of 2.
    \\

    \begin{algorithm}[]
        \begin{algorithmic}[1]
            \Function{MultiplyNDegreePolynomials}{}
            \State{} \Return{$0$}
            \EndFunction{}
        \end{algorithmic}
        \caption{Multiply \(n\) degree polynomial}
    \end{algorithm}
\end{homeworkProblem}

\pagebreak

\begin{homeworkProblem}
    Give an \(O(\log n)\) time algorithm that computes the following function:
    \\

    \alg{MEDIAN-OF-TWO($l_1, l_2$)}

    \begin{algorithmic}[1]
        \Require \(l_1\) and \(l_2\) are two sorted lists of integers. Each
        list has \(n\) elements (\(2n\) elements in total) and the value of
        each element in the lists is unique.

        \Ensure The value of the \(n^{th}\) smallest integer in the set of
        \(2n\) integers is \(l_1\) and \(l_2\).
    \end{algorithmic}
\end{homeworkProblem}

\pagebreak

\begin{homeworkProblem}
    Give an \(O(n)\) average case running time algorithm that computes the
    following:
    \\

    \alg{Kth-SMALLEST($list, k$)}

    \begin{algorithmic}[1]
        \Require An unsorted list \(list\) of unique integers and an integer \(k\)

        \Ensure The value of the \(k^{th}\) smallest integer from the list
    \end{algorithmic}
\end{homeworkProblem}

\end{document}
