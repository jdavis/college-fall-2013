\documentclass{article}

\usepackage{fancyhdr}
\usepackage{lastpage}
\usepackage{extramarks}
\usepackage[usenames,dvipsnames]{color}
\usepackage{amsmath}
\usepackage{amsthm}
\usepackage{amsfonts}
\usepackage{changepage}
\usepackage{lineno}
\usepackage{algpseudocode}

\usepackage{tikz}

\topmargin=-0.45in
\evensidemargin=0in
\oddsidemargin=0in
\textwidth=6.5in
\textheight=9.0in
\headsep=0.25in

\linespread{1.1}

\pagestyle{fancy}
\lhead{\hmwkAuthorName}
\chead{\hmwkClass\ (\hmwkClassInstructor\ \hmwkClassTime): \hmwkTitle}
\rhead{\firstxmark}
\lfoot{\lastxmark}
\cfoot{}
\renewcommand\headrulewidth{0.4pt}
\renewcommand\footrulewidth{0.4pt}

\setlength\parindent{0pt}

\newcommand{\enterProblemHeader}[1]{
    \nobreak\extramarks{}{Problem \arabic{#1} continued on next page\ldots}\nobreak{}
    \nobreak\extramarks{Problem \arabic{#1} (continued)}{Problem \arabic{#1} continued on next page\ldots}\nobreak{}
}

\newcommand{\exitProblemHeader}[1]{
    \nobreak\extramarks{Problem \arabic{#1} (continued)}{Problem \arabic{#1} continued on next page\ldots}\nobreak{}
    \stepcounter{#1}
    \nobreak\extramarks{Problem \arabic{#1}}{}\nobreak{}
}

\setcounter{secnumdepth}{0}
\newcounter{homeworkProblemCounter}
\setcounter{homeworkProblemCounter}{1}
\nobreak\extramarks{Problem \arabic{homeworkProblemCounter}}{}\nobreak{}

\newenvironment{homeworkProblem}{
    \section{Problem \arabic{homeworkProblemCounter}}
    \enterProblemHeader{homeworkProblemCounter}
}{
    \exitProblemHeader{homeworkProblemCounter}
}

\newcommand{\hmwkTitle}{Homework\ \#1}
\newcommand{\hmwkDueDate}{September 11, 2013 at 4:30pm}
\newcommand{\hmwkClass}{CS311}
\newcommand{\hmwkClassTime}{Section 3}
\newcommand{\hmwkClassInstructor}{Professor Lathrop}
\newcommand{\hmwkAuthorName}{Josh Davis}

\title{
    \vspace{2in}
    \textmd{\textbf{\hmwkClass:\ \hmwkTitle}}\\
    \normalsize\vspace{0.1in}\small{Due\ on\ \hmwkDueDate}\\
    \vspace{0.1in}\large{\textit{\hmwkClassInstructor\ \hmwkClassTime}}
    \vspace{3in}
}

\author{\textbf{\hmwkAuthorName}}
\date{}

\begin{document}

\maketitle

\pagebreak

\begin{homeworkProblem}
    Demonstrate that the quicksort implementation given is incorrect by
    providing an input on which the algorithm does not produce correctly sorted
    input. Briefly explain why the input fails.
    \\

    \textbf{Solution}

    The following input \(list = [3, 3]\), \(start = 0\), and \(end = 1\), will
    cause the given implementation of quicksort to loop infinitely.
    \\

    \textbf{Explanation}

    The reason this input fails is because when the first call to \(partition\)
    happens, \(left = start + 1 = 1\) and \(right = 1\). The \(while\) loop on
    line 3 waits until \(left \leq right\).
    \\

    However, \(left\) and \(right\) can never advance past each other because
    neither line 4 or 7 take into the account the possibility when the pivot
    value equals either \(list[left]\) or \(list[right]\).  To fix this, all we
    need to do is change line 4 to be less than \textbf{or equal}, \(list[left]
    \leq list[start]\). Now \(left\) can be incremented and cause the \(while\)
    loop to finish.
    \\

    There is one last issue. When \(partition\) returns, it will set \(mid
    = 1\). The problem though is that we recursively call \(quicksort\) on
    \(start = 0\) and \(end = 1\). Which is the exact same as the first call
    that we made to \(quicksort\).  To fix this, the first call to
    \(quicksort\) should pass \(mid - 1\) instead of \(mid\).
    \\

    This is what the final code should look like:

    \begin{linenumbers}[1]
        \renewcommand\linenumberfont{\normalfont\bfseries\small}
        \begin{algorithmic}
            \Function{Quick-Sort}{$list, start, end$}
                \If{$start \geq end$}
                    \State{} \Return{}
                \EndIf{}
                \State{} $mid \gets \Call{Partition}{list, start, end}$
                \State{} \Call{Quick-Sort}{$list, start, mid - 1$}
                \State{} \Call{Quick-Sort}{$list, mid + 1, end$}
            \EndFunction{}
        \end{algorithmic}
    \end{linenumbers}

    and:

    \begin{linenumbers}[1]
        \renewcommand\linenumberfont{\normalfont\bfseries\small}
        \begin{algorithmic}
            \Function{Partition}{$list, start, end$}
                \State{} $left \gets start + 1$
                \State{} $right \gets end$
                \While{$left \leq right$}
                    \While{$left \leq end \wedge list[left] \leq list[start]$}
                        \State{} $left \gets left + 1$
                    \EndWhile{}
                    \While{$right \geq start \wedge list[right] > list[start]$}
                        \State{} $right \gets right - 1$
                    \EndWhile{}
                    \If{$left \leq right$}
                        \State{} \Call{swap}{$list, left, right$}
                    \EndIf{}
                \EndWhile{}
                \State{} \Call{swap}{$list, start, right$}
                \State{} \Return{} $right$
            \EndFunction{}
        \end{algorithmic}
    \end{linenumbers}

\end{homeworkProblem}

\pagebreak

\begin{homeworkProblem}


    \textbf{Part A}

    \begin{proof}
    \end{proof}

    \textbf{Part B}

    \begin{proof}
    \end{proof}

    \textbf{Part C}

    \begin{proof}
    \end{proof}

    \textbf{Part D}

    \begin{proof}
    \end{proof}
\end{homeworkProblem}

\pagebreak

\begin{homeworkProblem}
    Using the implemenation of insertion sort given, solve the given problems.
    \\

    \textbf{Part A}

    Identify a useful invariant \(I_0\) for the outer loop, that is, a loop
    invariant which will allow you to complete parts \(b\) through \(d\).

    \textbf{Solution}

    \textbf{Part B}

    Prove that if \(I_0\) holds after the last iteration of the outer loop
    completes, then the algorithm sorts \(list\) in ascending order. This
    corresponds to the termination step of the loop invariant proof.

    \begin{proof}
    \end{proof}

    \textbf{Part C}

    Prove that \(I_0\) holds before the loop is executed for the first time.
    This corresponds to the initialization step of the loop invariant proof.

    \begin{proof}
    \end{proof}

    \textbf{Part D}

    Prove that \(I_0\) is an invariant for the outer loop.

    \textbf{Hint:} It will be necessary to prove something about the behavior
    of the inner loop using a loop invariant as well. This corresponds to the
    maintenance step of the loop invariant.

    \begin{proof}
    \end{proof}
\end{homeworkProblem}

\begin{homeworkProblem}
    Prove or disprove:

    \[
        (A - B) \cup C = (A \cup B) - C
    \]

    \begin{proof}

    To prove that \((A - B) \cup C = (A \cup B) - C\), we must show that \((A -
    B) \cup C \subseteq (A \cup B) - C\) and that \((A \cup B) - C \subseteq (A
    - B) \cup C\).

    \textbf{Part One}

    First we will show that \((A - B) \cup C \subseteq (A \cup B) - C\).


    \textbf{Part Two}

    Now we will show that \((A \cup B) - C \subseteq (A - B) \cup C\).
    \\

    Since we have shown each sides, our proof is complete and we have shown
    that \((A - B) \cup C = (A \cup B) - C\).

    \textbf{Alternative Solution}

    \[
        \begin{split}
            (A - B) \cup C &= (\overline{A} \cap B) \cup C
            \\
            &= (\overline{A} \cap B) \cup C
            \\
            &= \overline{(A \cup \overline{B})} \cup C
            \\
            &= \overline{(A \cup B)} \cup C
            \\
            &= (A \cup B) - C
        \end{split}
    \]

    \end{proof}
\end{homeworkProblem}

\begin{homeworkProblem}
    Given an instance of the longest common subsequence problem.

    \textbf{Solution}

    Given two strings, \(s = ``abcdefg"\) and \(t = ``biczdxyg"\), the longest
    common subsequence between the two is \(``bcdg"\).
\end{homeworkProblem}

\pagebreak

\begin{homeworkProblem}
    One hundred quarters lie scattered about on a table before you. Some known
    n of them are heads up, while the rest are tails up. You cannot in any way
    observe the configuration of individual quarters (perhaps you are
    blindfolded). You can, however, pick them up and place them back on the
    table in the opposite configuration, ie flip them.
    \\

    Your objective is to
    divide the one hundred quarters into two groups such that each group has
    the same number of heads. The two groups need not have the same number of
    quarters. How do you do this?
\end{homeworkProblem}

\end{document}
